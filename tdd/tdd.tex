\documentclass{beamer}
\usepackage{tikz}
\usepackage{pdfpages}
\usepackage{animate}

\usetheme{Warsaw}
% or ...

% \setbeamercovered{transparent}
% or whatever (possibly just delete it)


\usepackage[english]{babel}
% or whatever

\usepackage[utf8]{inputenc}
% or whatever

\usepackage[T1]{fontenc}
% Or whatever. Note that the encoding and the font should match. If T1
% does not look nice, try deleting the line with the fontenc.


\title{Three-dimensional straight-line drawings of planar graphs} % (optional, use only with long paper titles)

% \subtitle
% {Presentation Subtitle} % (optional)

\author
{Tobias Feigenwinter}

\institute
{
ETH Zürich}
% - Use the \inst command only if there are several affiliations.
% - Keep it simple, no one is interested in your street address.

\date % (optional)
{2023/04/21}

\subject{Talks}
% This is only inserted into the PDF information catalog. Can be left
% out. 



% If you have a file called "university-logo-filename.xxx", where xxx
% is a graphic format that can be processed by latex or pdflatex,
% resp., then you can add a logo as follows:

% \pgfdeclareimage[height=0.5cm]{university-logo}{university-logo-filename}
% \logo{\pgfuseimage{university-logo}}



% Delete this, if you do not want the table of contents to pop up at
% the beginning of each subsection:
\AtBeginSubsection[]
{
    \begin{frame}<beamer>{Outline}
	\tableofcontents[currentsection,currentsubsection]
    \end{frame}
}


% If you wish to uncover everything in a step-wise fashion, uncomment
% the following command: 

%\beamerdefaultoverlayspecification{<+->}


\newcommand\still[2][1]{
    \includegraphics[page=#1]{#2.pdf}
}

\newcommand\footcite[1]{%
  \begingroup
  \renewcommand\thefootnote{}\footnote{#1}%
  \addtocounter{footnote}{-1}%
  \endgroup
}
\begin{document}
\begin{frame}
    \titlepage
\end{frame}

\begin{frame}{Outline}
    \tableofcontents[pausesections]
 % You might wish to add the option [pausesections]
\end{frame}

\section{Motivation for 3D drawings}
\section{b}
\begin{frame}{Why 3D drawings?}{}
    \begin{itemize}
	\item<+->
	    Nice-looking and easy-to-understand graphs

	    \only<+> {
		\still{static/icosahedralgraph}
	    }
	    \only<+> {
		\still{static/cube}
	    }
	    \only<+> {
		\still{anim/icosahedron}
	    }
	    \only<+> {
		\animategraphics[autoplay, loop]{20}{anim/icosahedron}{}{}
	    }
	    \only<+> {
		\still{anim/rand3d}
	    }
	    \only<+> {
		\animategraphics[autoplay, loop]{20}{anim/rand3d}{}{}
	    }
	\item<+->
	    User Interfaces 
	    \only<+>{
		\includegraphics[height=0.6\paperheight, page=5, trim={0 17cm 0 0}, clip]{figures/Conetrees.5.pdf}
		\footcite{Figure from: George, Jock and Stuart (1991): \textit{Cone Trees: Animated 3D Visualizations of Hierarchical Information}}
	    }
	    \only<+->{(Virtual Reality?)}
    \end{itemize}
\end{frame}

\begin{frame}{Main result}
    \footcite{Theorem from: Dujmović, Joret, Micek, Morin, Ueckerdt and Wood (2020): \textit{Planar Graphs Have Bounded Queue Number}}
    \begin{theorem}
	Every planar graph with $n$ vertices has a 3-dimensional straight-line drawing \pause on the integer grid \pause with $\mathcal O(n)$ Volume.
    \end{theorem}
    \only<1>{
	\animategraphics[autoplay, loop]{20}{anim/rand3d}{}{}
    }
    \only<2-3>{
	\animategraphics[autoplay, loop]{20}{anim/gridgraphexample}{}{}
    }
\end{frame}

\begin{frame}{Make Titles Informative.}

    You can create overlays\dots
    \begin{itemize}
	\item using the \texttt{pause} command:
	    \begin{itemize}
		\item
		    First item.
		    \pause
		\item    
		    Second item.
	    \end{itemize}
	\item
	    using overlay specifications:
	    \begin{itemize}
		\item<3->
		    First item.
		\item<4->
		    Second item.
	    \end{itemize}
	\item
	    using the general \texttt{uncover} command:
	    \begin{itemize}
		    \uncover<5->{\item
		    First item.}
		    \uncover<6->{\item
		    Second item.}
	    \end{itemize}
    \end{itemize}
\end{frame}


\subsection{Second Subsection}

\begin{frame}{Make Titles Informative.}
\end{frame}

\begin{frame}{Make Titles Informative.}
\end{frame}



\section*{Summary}

\begin{frame}{Summary}

  % Keep the summary *very short*.
    \begin{itemize}
	\item
	    The \alert{first main message} of your talk in one or two lines.
	\item
	    The \alert{second main message} of your talk in one or two lines.
	\item
	    Perhaps a \alert{third message}, but not more than that.
    \end{itemize}

  % The following outlook is optional.
    \vskip0pt plus.5fill
    \begin{itemize}
	\item
	    Outlook
	    \begin{itemize}
		\item
		    Something you haven't solved.
		\item
		    Something else you haven't solved.
	    \end{itemize}
    \end{itemize}
\end{frame}


\end{document}
