\documentclass[t]{beamer}
\usepackage{tikz}
\usepackage{pdfpages}
\usepackage{animate}

\usetheme{Montpellier}
% or ...
% \usecolortheme{dove}
\useoutertheme{infolines}

% \setbeamercovered{transparent}
% or whatever (possibly just delete it)


\usepackage[english]{babel}
% or whatever

\usepackage[utf8]{inputenc}
% or whatever

\usepackage[T1]{fontenc}
% Or whatever. Note that the encoding and the font should match. If T1
% does not look nice, try deleting the line with the fontenc.


\title{Three-dimensional straight-line drawings of planar graphs} % (optional, use only with long paper titles)

% \subtitle
% {Presentation Subtitle} % (optional)

\author
{Tobias Feigenwinter}

\institute
{
ETH Zürich}
% - Use the \inst command only if there are several affiliations.
% - Keep it simple, no one is interested in your street address.

\date % (optional)
{2023/04/21}

\subject{Talks}
% This is only inserted into the PDF information catalog. Can be left
% out. 



% If you have a file called "university-logo-filename.xxx", where xxx
% is a graphic format that can be processed by latex or pdflatex,
% resp., then you can add a logo as follows:

% \pgfdeclareimage[height=0.5cm]{university-logo}{university-logo-filename}
% \logo{\pgfuseimage{university-logo}}



% Delete this, if you do not want the table of contents to pop up at
% the beginning of each subsection:
\AtBeginSubsection[]
{
    \begin{frame}<beamer>{Outline}
	\tableofcontents[currentsection,currentsubsection]
    \end{frame}
}


% If you wish to uncover everything in a step-wise fashion, uncomment
% the following command: 

%\beamerdefaultoverlayspecification{<+->}


\newcommand\still[2][1]{
    \includegraphics[page=#1]{#2.pdf}
}

\newcommand\footcite[1]{%
  \begingroup
  \renewcommand\thefootnote{}\footnote{#1}%
  \addtocounter{footnote}{-1}%
  \endgroup
}

\newtheorem{task}{Task}
\newtheorem{claim}{Claim}
\begin{document}
\begin{frame}
    \titlepage
\end{frame}

\begin{frame}{Outline}
    \tableofcontents[pausesections]
 % You might wish to add the option [pausesections]
\end{frame}

\section{Motivation for 3D drawings}
\begin{frame}{Why 3D drawings?}{}
    \begin{itemize}
	\item<+->
	    Nice-looking and easy-to-understand graphs

	    \only<+> {
		\still{static/icosahedralgraph}
	    }
	    \only<+> {
		\still{static/cube}
	    }
	    \only<+> {
		\still{anim/icosahedron}
	    }
	    \only<+> {
		\animategraphics[autoplay, loop]{20}{anim/icosahedron}{}{}
	    }
	    \only<+> {
		\still{anim/rand3d}
	    }
	    \only<+> {
		\animategraphics[autoplay, loop]{20}{anim/rand3d}{}{}
	    }
	\item<+->
	    User Interfaces 
	    \only<+>{
		\includegraphics[height=0.6\paperheight, page=5, trim={0 17cm 0 0}, clip]{figures/Conetrees.5.pdf}
		\footcite{Figure from: George, Jock and Stuart (1991): \textit{Cone Trees: Animated 3D Visualizations of Hierarchical Information}}
	    }
	    \only<+->{(Virtual Reality?)}
    \end{itemize}
\end{frame}

\begin{frame}{Main result}
    \footcite{Theorem from: Dujmović, Joret, Micek, Morin, Ueckerdt and Wood (2020): \textit{Planar Graphs Have Bounded Queue Number}}
    \begin{theorem}
	\uncover<+->{Every planar graph with $n$ vertices has a 3-dimensional straight-line drawing on the integer grid}
	\uncover<+->{with $\mathcal O(n)$ Volume.}
    \end{theorem}
    \begin{center}
	\animategraphics[autoplay, loop]{20}{anim/gridgraphexample}{}{}
    \end{center}
\end{frame}

\begin{frame}{Why is this interesting?}
\end{frame}

\section{Results for general graphs}

\begin{frame}{The Moment Curve}
    \footcite{Theorem and proof from Cohen, Eades, Tao Lin and Ruskey: \textit{Three-Dimensional Graph Drawing}}
    \uncover<+->{
	\begin{theorem}
	    Every graph with $n$ vertices has a 3-dimensional straight-line drawing 
	    on the integer grid
	    with $\mathcal O(n^6)$ Volume.
	\end{theorem}
    }
    \begin{proof}<+->
	\begin{itemize}
	    \item<.-> Draw vertex $v$ at $(v, v^2, v^3)$
		\only<.>{
		    \begin{center}
			\animategraphics[autoplay, loop]{2}{anim/momentcurve}{}{}
		    \end{center}
		}
	    \item<+-> No four points on the Moment Curve are coplanar
		\only<+>{
		    \begin{itemize}
			\item Four points $(x_i, y_i, z_i)$ for $i\in\left\{ 0, 1, 2, 3 \right\}$ are coplanar iff 
			    \begin{equation*}
				\det\left(
				\begin{matrix}
				    1&x_0&y_0&z_0\\
				    1&x_1&y_1&z_1\\
				    1&x_2&y_2&z_2\\
				    1&x_3&y_3&z_3\\
				\end{matrix}
			    \right) = 0
			    \end{equation*}
		    \end{itemize}
		}
		\only<+-+(+1)>{
		    \begin{itemize}
			\item Four points $(v_i, v_i^2, v_i^3)$ for $i\in\left\{ 0, 1, 2, 3 \right\}$ are coplanar iff 
			    \begin{equation*}
				\det\left(
				\begin{matrix}
				    1&v_0&v_0^2&v_0^3\\
				    1&v_1&v_1^2&v_1^3\\
				    1&v_2&v_2^2&v_2^3\\
				    1&v_3&v_3^2&v_3^3\\
				\end{matrix}
			    \right) = 0
			    \end{equation*}
			\item<+> {
				This is the Vandermonde Matrix with determinant $\prod_{0 \le i < j < 4} (v_j - v_i) \neq 0$
			}
		    \end{itemize}
		}
	    \item<+-> No four vertices are coplanar $\implies$ No two edges cross
	\end{itemize}
    \end{proof}
\end{frame}

\begin{frame}{The Modular Moment Curve}
    \footcite{Theorem and proof from Cohen, Eades, Tao Lin and Ruskey: \textit{Three-Dimensional Graph Drawing}}
    \uncover<+->{
	\begin{theorem}
	    Every graph with $n$ vertices has a 3-dimensional straight-line drawing 
	    on the integer grid
	    with $\mathcal O(n^{\alert<.>{3}})$ Volume.
	\end{theorem}
    }
    \begin{proof}<+->
	\begin{itemize}
	    \item<.-> Draw vertex $v$ at $(v, v^2\alert<.>{\mod p}, v^3\alert<.>{\mod p})$ \alert<.>{with $n \le p < 2n$, $p$ prime}
		\only<.>{
		    \begin{center}
			\animategraphics[autoplay, loop]{20}{anim/modularmomentcurve}{}{}
		    \end{center}
		}
		\only<+>{
		    \begin{center}
			\animategraphics[autoplay, loop]{20}{anim/modularmomentcompletegraph}{}{}
		    \end{center}
		}
	    \item<+-> No four such vertices are coplanar
		\only<+-+(+3)>{
		    \begin{itemize}
			\item The vertices $v_0, v_1, v_2, v_3$ are coplanar iff 
			    \begin{equation*}
				\det\left(
				\begin{matrix}
				    1&v_0&v_0^2\alert<.>{\mod p}&v_0^3\alert<.>{\mod p}\\
				    1&v_1&v_1^2\alert<.>{\mod p}&v_1^3\alert<.>{\mod p}\\
				    1&v_2&v_2^2\alert<.>{\mod p}&v_2^3\alert<.>{\mod p}\\
				    1&v_3&v_3^2\alert<.>{\mod p}&v_3^3\alert<.>{\mod p}\\
				\end{matrix}
			    \right) = 0
			    \end{equation*}
			\item<+-+(+2)> {
				This determinant is $\alert<.>{\equiv} \prod_{0 \le i < j < 4} \alt<+>{\underbrace{(v_j-v_i)}_{\not\equiv0}}{(v_j - v_i)} \uncover<+>{\alert<.>{\not\equiv} 0} \alert<.(-2)>{\pmod p}$
			}
		    \end{itemize}
		}
	    \item<+-> No four vertices are coplanar $\implies$ No two edges cross
	\end{itemize}
    \end{proof}
\end{frame}

\begin{frame}{$\mathcal O(n^3)$ is tight for complete graphs}
    \footcite{Theorem and proof from Cohen, Eades, Tao Lin and Ruskey: \textit{Three-Dimensional Graph Drawing}}
    \begin{itemize}
	\item Cut grid into planes \pause
	\item At most four vertices per plane\pause
	\item At least $\lceil\frac{n}{4}\rceil$ layers in each dimension\pause
	\item At least $\lceil\frac{n}{4}\rceil^{3} \in \Omega(n^3)$ Volume
    \end{itemize}
    \only<*>{
	\begin{center}
	    \animategraphics[autoplay, loop]{20}{anim/upperboundgrid}{}{}
	\end{center}
    }
\end{frame}

\section{Proof of the main result}

\subsection{Track Layouts to 3D drawings}

\begin{frame}{Track Layouts to 3D drawings}
    \begin{theorem}
	Every graph on $n$ vertices with track number $t$ has a 3-dimensional straight-line drawing on the integer grid with $\mathcal O(t^3n)$ volume.
    \end{theorem}
\end{frame}

\begin{frame}{Track Layouts}
    \foreach \i in {1, ..., 4} {
	\only<+>{
	    \begin{center}
		\still[\i]{static/tracklayout}
	    \end{center}
	}
    }
    \only<+->{
    \begin{definition}[Track Layout]
	A \textit{$t$-Track Layout} of some graph $G$ consists of
	\begin{itemize}
	    \item<+-> a partition of the vertices of $G$ into independent sets $\left\{ T_i \right\}_{i\in\left\{ 0, 1, \ldots, t-1 \right\}}$, called tracks, and
	    \item<+-> an ordering $<_i$ of the vertices of each track $T_i$.
	\end{itemize}
	\pause
	It does not contain an X-Crossing. \pause An \textit{X-Crossing} consists of two edges $vw$ and $xy$ between two tracks $T_i \ni v,x$ and $T_j \ni w, y$ such that $v <_i x$ and $w >_j y$
    \end{definition}
    \pause
    \begin{definition}[Track Number]
	The \textit{Track Number} $\operatorname{tn}(G)$ is the smallest $t$ such that $G$ has a $t$-track layout
    \end{definition}
}
\end{frame}

\begin{frame}{The Modular Moment Curve with tracks}
    % TODO: Incomplete, blackboard?
    \footcite{Theorem and proof from Cohen, Eades, Tao Lin and Ruskey: \textit{Three-Dimensional Graph Drawing}}
    \uncover<+->{
	\begin{theorem}
	    Every graph on $n$ vertices with track number $t$ has a 3-dimensional straight-line drawing on the integer grid with $\mathcal O(t^3n)$ volume.
	\end{theorem}
    }
    \begin{proof}<+->
	\begin{itemize}
	    \item<.-> Draw the $j$-th vertex of the track $T_j$ at $(i, i^2\mod p, i^3\mod p + jp)$ with $n \le p < 2n$, $p$ prime
		\only<+>{
		    \begin{center}
			\animategraphics[autoplay, loop]{20}{anim/trackmomentcurve}{}{}
		    \end{center}
		}
		\only<.>{
		    \begin{center}
			\animategraphics[autoplay, loop]{20}{anim/trackmomentcurve}{}{}
		    \end{center}
		}
	    \item<+-> No four such vertices are coplanar
		\only<+-+(+3)>{
		    \begin{itemize}
			\item The vertices $v_0, v_1, v_2, v_3$ are coplanar iff 
			    \begin{equation*}
				\det\left(
				\begin{matrix}
				    1&v_0&v_0^2\alert<.>{\mod p}&v_0^3\alert<.>{\mod p}\\
				    1&v_1&v_1^2\alert<.>{\mod p}&v_1^3\alert<.>{\mod p}\\
				    1&v_2&v_2^2\alert<.>{\mod p}&v_2^3\alert<.>{\mod p}\\
				    1&v_3&v_3^2\alert<.>{\mod p}&v_3^3\alert<.>{\mod p}\\
				\end{matrix}
			    \right) = 0
			    \end{equation*}
			\item<+-+(+2)> {
				This determinant is $\alert<.>{\equiv} \prod_{0 \le i < j < 4} \alt<+>{\underbrace{(v_j-v_i)}_{\not\equiv0}}{(v_j - v_i)} \uncover<+>{\alert<.>{\not\equiv} 0} \alert<.(-2)>{\pmod p}$
			}
		    \end{itemize}
		}
	    \item<+-> No four vertices are coplanar $\implies$ No two edges cross
	\end{itemize}
    \end{proof}
\end{frame}

\section{Queues to Tracks}

\begin{frame}
    \begin{theorem}
	Every graph on $n$ vertices with queue number $q$ and acyclic chromatic number $c$ has track number $c(2q)^{c-1}$
    \end{theorem}
\end{frame}

\begin{frame}
    \begin{definition}[Acyclic coloring]
	An acyclic coloring is a proper coloring such that every bichromatic subgraph is a forest, or equivalently, such that every cyclie receives at least three colors. 
    \end{definition}
    \pause
    \begin{definition}[Acyclic Chromatic Number]
	The acyclic chromatic number $\chi_a(G)$ is the smallest number of colors across all acyclic colorings of $G$.
    \end{definition}
    \foreach\i in {1, ..., 5}{
	\only<+>{
	    \begin{center}
		\still[\i]{static/acycliccoloring}
	    \end{center}
	}
    }
\end{frame}

\begin{frame}
    \begin{columns}
	\begin{column}{0.4\textwidth}
	    \begin{lemma}[Refinement Lemma]
		Given a graph $G$ with
		\begin{itemize}
		    \item an acyclic {\only<3->{\relax}$c$}-coloring
		    \item a (not necesserily proper) {\only<3->{\relax}$k$}-edge-coloring,
		\end{itemize}
		there is a proper $ck^{c-1}$-vertex-coloring of $G$ such that
		\begin{itemize}
		    \item every new vertex color class is contained in one old vertex color class, \rlap{and}
		    \item the edges between any two vertex color classes are monochromatic. 
		\end{itemize}
	    \end{lemma}
	\end{column}
	\begin{column}{0.6\textwidth}%
	    \vspace{-0.5cm}
	    \foreach\i in {1, ..., 15}{%
		\only<+>{%
		    \begin{center}%
			\still[\i]{static/refinementlemma}%
		    \end{center}%
		}%
	    }%
	\end{column}%
    \end{columns}
\end{frame}

\begin{frame}
    \begin{theorem}
	Every graph on $n$ vertices with queue number $q$ and acyclic chromatic number $c$ has track number $c(2q)^{c-1}$
    \end{theorem}
    \begin{proof}
	\begin{itemize}
		\pause
	    \item Let $\le_Q$ be the linear ordering from a $q$-queue-layout. 
		\pause
	    \item Use the $q$ queues for a $q$-edge-coloring.
		\pause
	    \item Apply the refinement lemma to an acyclic $c$-coloring and the above edge $q$-coloring.
		\pause
	    \item The tracks are the refined colors, ordered by $\le_Q$
	\end{itemize}
    \end{proof}
	\pause
	Do the ``tracks'' actually form a track layout? \pause 
	\begin{itemize}
	    \item NO!
		\pause
	    \item But we can fix them
	\end{itemize}
\end{frame}

\begin{frame}
    \begin{theorem}
	Every graph on $n$ vertices with queue number $q$ and acyclic chromatic number $c$ has track number at most $c(2q)^{c-1}$
    \end{theorem}
    \begin{proof}
	\begin{itemize}
		\pause
	    \item Let $\le_Q$ be the linear ordering from a $q$-queue-layout. 
		\pause
		    \only<-8> {%
		    \item Use the $q$ queues for a $q$-edge-coloring.%
		    }%
		    \only<9-> {%
			\alert<9>{%
		    \item Partition the edges of each queue into forward and backward edges, then use the resulting $2q$ sets as a $2q$-edge-coloring.%
		    }%
		    }%
		\pause
	    \item Apply the refinement lemma to an acyclic $c$-coloring and the above edge \alert<9>{$2q$}-coloring.
		\pause
	    \item The tracks are the refined colors, ordered by $\le_Q$
	\end{itemize}
	\only<9->{
	    Do the ``tracks'' actually form a track layout?
	    \begin{itemize}
		\item<10-> YES!
	\end{itemize}
	    }
    \end{proof}
    \pause
    \only<-8>{
	Do the ``tracks'' actually form a track layout? \pause 
	\begin{itemize}
	    \item NO!
		\pause
	    \item But we can fix them
    \end{itemize}
	    }
\end{frame}

\section{Acyclic chromatic number}
\begin{frame}
    \begin{theorem}
	Outerplanar graphs have acyclic chromatic number at most three
    \end{theorem}
    \begin{proof}
	Let $G$ be an outerplanar graph. WLoG assume $G$ is a triangulated polygon. Recursively color the graph as follows:
	\begin{itemize}
	    \item Choose $v$ with degree $2$ and neighbors $u$ and $w$.
	    \item Color $G\setminus v$.
	    \item Color $v$ using the color $u$ and $w$ don't use. 
	    \item The addition of $v$ can't create a new bichromatic cycle: Otherwise, $u$ and $w$ would have the same color, but they are neighbors.
	\end{itemize}
    \end{proof}
\end{frame}

\begin{frame}
    \begin{theorem}
	Planar graphs have acyclic chromatic number at most nine.
    \end{theorem}
    \begin{proof}
	Let $G$ be a planar graph. WLOG assume $G$ is connected. We will construct a $9$-coloring and show that it does not contain a bichromatic cycle (and that it is proper).
	\begin{itemize}
	    \item Choose some $v_0$.
	    \item We define the \emph{circle} $C_i = \left\{v \in G | d(v, v_0)=i\right\}$. Similarly define $C_{<i}$ etc.
	    \item Use a different set of three colors for the circles $C_{3k}$, $C_{3k+1}$, $C_{3k+2}$
	    \item Observe: Edges of $G$ are within a single circle or between two adjacent circles. Any bichromatic cycle can use (up to) two adjacent circles. 
	    \item We will now color each circle $S_i$ such that it cannot be the innermost shell of a bichromatic cycle.
	\end{itemize}
	\phantom\qedhere
    \end{proof}
\end{frame}

\begin{frame}{Coloring the circles}
    \begin{task}
	Color the circle $C_i$ such that it cannot be the innermost shell of a bichromatic cycle.
    \end{task}
    \begin{itemize}
	\item Start with the induced subgraph $G[C_i]$. 
	\item For each vertex $v \in C_{j+1}$, let $v_1, v_2, \dots, v_j$ be the cyclic order of the vertices in $C_i$ adjacent to $v$. Add edges $v_1v_2, v_2v_3, \dots, v_jv_1$ if they don't exist. 
    \end{itemize}
    We can see the resulting graph $C_i^*$ is outerplanar if we:
    \begin{itemize}
	\item Notice that $G[C_{<i}]$ is connected and each vertex of $C_i$ has an edge into $C_{<i}$ (in $G$)
	\item Follow that the vertices of $C_i$ belong to a common face in $G[C_{\ge i}]$
	\item Draw the edges added to form $C_i^*$ by closely following the edges from the vertex $v \in C_{i+1}$ of the current step.
    \end{itemize}
    So, we can color $C_i^*$ acyclically with three colors.
\end{frame}

\begin{frame}
    \begin{claim}
	$C_i$, colored as described above, cannot be the innermost circle of a bichromatic cycle of $G$
    \end{claim}
    \begin{proof}
	Assume for a contradiction that such a cycle exists. This cycle consists of alternating vertices of $C_i$ and $C_{i+1}$:

	Let $v_0v_1v_2$ be consecutive vertices on the cycle with $v_1\in C_{i+1}$. We distinguish by the degree of $v_1$
	\begin{itemize}
	    \item Degree $1$: Impossible, because $v_0, v_2$ are both neighbors of $v_1$
	    \item Degree $2$ or $3$: Impossible, because $v_0, v_2$ are neighbors in $C_{i}^*$
	    \item Degree $4$ or more: Take the vertices to the left and right of $v_0$ in cyclic ordering around $v_1$. They, together with $v_1$ and $C_{<i}$, cut off $v_0$ (or $v_1$) from the other end of the cycle. 
	\end{itemize}
    \end{proof}
\end{frame}

\begin{frame}{Concluding the proof}
    \begin{theorem}
	Planar graphs have acyclic chromatic number at most nine.
    \end{theorem}
    \begin{proof}[Proof (cont.)]
	We have shown that, using our coloring, no bichromatic cycle exists. It is easy to see that the coloring is proper. Therefore, we have found an acyclic $9$-coloring.
    \end{proof}
\end{frame}

\begin{frame}{Better bounds}
    \begin{theorem}
	Planar graphs have acyclic chromatic number at most five.
    \end{theorem}
    \begin{proof}
	Same as our proof for outerplanar graphs (i.e. reducing any graph to another graph with less vertices), except with 450 cases.{\tiny Details left as exercise.}
    \end{proof}
    \begin{theorem}
	Some planar graphs have acyclic chromatic number five.
    \end{theorem}
\end{frame}

\begin{frame}{We're done!}
    \begin{table}
	\centering
	\begin{tabular}{llll}
&Queue Number&Acyclic Chromatic Number&Track Number&3D Drawing Dimensions\\
Proven now      \\
Proven anytime  \\
Best lower bound\\
	    Queue Number & 42 \\
	    Acyclic Chromatic Number & 9\\
	    Track Number & $8 \cdot 10^{13}$\\
	    3D Drawing Volume & $16 \cdot 10^{13} \times 16\cdot 10^{13} \times n$\\
	\end{tabular}
	\caption{<+Caption text+>}
	\label{tab:<+label+>}
    \end{table}<++>
\end{frame}

\begin{frame}{Make Titles Informative.}
    You can create overlays\dots
    \begin{itemize}
	\item using the \texttt{pause} command:
	    \begin{itemize}
		\item
		    First item.
		    \pause
		\item    
		    Second item.
	    \end{itemize}
	\item
	    using overlay specifications:
	    \begin{itemize}
		\item<3->
		    First item.
		\item<4->
		    Second item.
	    \end{itemize}
	\item
	    using the general \texttt{uncover} command:
	    \begin{itemize}
		    \uncover<5->{\item
		    First item.}
		    \uncover<6->{\item
		    Second item.}
	    \end{itemize}
    \end{itemize}
\end{frame}


\subsection{Second Subsection}

\begin{frame}{Make Titles Informative.}
\end{frame}

\begin{frame}{Make Titles Informative.}
\end{frame}



\section*{Summary}

\begin{frame}{Summary}

  % Keep the summary *very short*.
    \begin{itemize}
	\item
	    The \alert{first main message} of your talk in one or two lines.
	\item
	    The \alert{second main message} of your talk in one or two lines.
	\item
	    Perhaps a \alert{third message}, but not more than that.
    \end{itemize}

  % The following outlook is optional.
    \vskip0pt plus.5fill
    \begin{itemize}
	\item
	    Outlook
	    \begin{itemize}
		\item
		    Something you haven't solved.
		\item
		    Something else you haven't solved.
	    \end{itemize}
    \end{itemize}
\end{frame}


\end{document}
